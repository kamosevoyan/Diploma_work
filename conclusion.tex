%Set the document class and font size
\documentclass[fleqn, bachelor,subf,12pt,notitlepage]{article}
\usepackage[utf8]{inputenc}
\usepackage{enumerate}
\usepackage{amsmath}
\usepackage{mathtools} 
\usepackage{amssymb}
\usepackage{systeme}
\usepackage[english]{babel}
\usepackage{xparse}
\usepackage{xfrac}
\usepackage{setspace}
\usepackage{multicol}
\usepackage{array}
\usepackage{tabularx}
\usepackage{bigints}
\usepackage{fontspec}
\usepackage{chngcntr}
\usepackage{caption}

%This package allows to modify enumerations
\usepackage{enumitem}


%This package allows to change figure insertion mode (H, etc.)
\usepackage{float}

%This package is used for big sum symbol
\usepackage{relsize}

\usepackage
[
 	a4paper,
 	left=30mm,
	right = 10mm,
 	top=20mm,
	bottom=25mm
 ]
{geometry}

%This command sets the font
\setmainfont{Sylfaen}

%These commands change line spacing
%\onehalfspacing
\linespread{1.5}

\title{Դիպլոմային աշխատանք}
\author{Կամո Սևոյան}

%This command allows to change page counter.
\setcounter{page}{6}

\begin{document}

\newpage
\section*{\centering {\addfontfeatures{FakeBold=2.0}Եզրակացություն}}
\sloppy

Այսպիսով, դիպլոմային աշխատանքում քննարկվել են մինչև երեք փոփոխականի ֆունկցիաների մոտարկման եղանակները, ավելի կոնկրետ Լագրանժի և էրմիթյան մոտարկման եղանակները, հատկապես ուշադրություն դարձնելով երկու փոփոխականի ֆունկցիաներին։ Կառուցված մոտարկման բանաձևերի օգնությամբ իրականացվել են մասնակի ածանցյալներով դիֆերենցիալ հավասարումների մոտավոր լուծումներ վարիացիոն մեթոդով։ Իրականացման տեսանկյունից հատկապես բարդ է Արգիրիսի բազային ֆունկցիաները, քանսզի դեպի ստանդարտ էլեմենտ և հակադարձ ձևափոխությունները բավականին բարդ են ի համեմատ մյուս բազիսային էլեմենտների։


\newpage
\section*{\centering {\addfontfeatures{FakeBold=2.0}Գրականության ցանկ}}



\end{document}